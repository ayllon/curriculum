\documentclass[11pt,a4paper]{moderncv}

% moderncv themes
\moderncvtheme[green]{classic}

% character encoding
\usepackage[utf8]{inputenc}

% Increase the title size
%\setlength{\makecvtitlenamewidth}{5cm}
\renewcommand*{\namefont}{\fontsize{30}{40}\mdseries\upshape}

% adjust the page margins
\usepackage[scale=0.8]{geometry}
\recomputelengths                             % required when changes are made to page layout lengths

\usepackage{footmisc}


% hyphenation
\hyphenation{Go-vern-ment Ayllón}

% personal data
\firstname{Alejandro}
\familyname{Álvarez Ayllón}
\title{Software Engineer}
\address{27 Promenade de l'Europe}{1203 Geneva (Switzerland)}
\phone{(+41) 076 237 6637}
\email{a.alvarezayllon@gmail.com}
\social[linkedin]{aalvarezayllon}

\begin{document}
\maketitle


\section{Profile}
I am a Software Engineer with more than ten years of experience in C++.
I started in 2010 with the re-design and implementation in C++03 of one of the LHC Grid storage components,
a design that is still in use today. During my career, I have switched from C++03 to C++11 and finally C++14.
I enjoy the language more with each version, and I look forward to being able to use more recent standards in
my everyday work, although I would be more than happy to be given a chance to work with others.
I take great pride in my job for all the projects I have worked on, technically and otherwise.
I consider it essential to keep in mind the final objectives --- software is the means to an end ---
and always have a respectful attitude towards users, colleagues, and even code!
It is, after all, someone else's work.


\section{Work experience}
\cventry{March 2018 -- Present}{Programmer Analyst}{Department of Astronomy, University of Geneva}{Geneva}{Switzerland}{
As a member of the Swiss Science Data Center inside the Euclid Consortium, I participate in the
development, maintenance, testing, and optimization of the software used to determine the distance
of galaxies --- red-shift --- using photometric measurements.\\
I also work on the astronomical image processing suite \emph{SourceXtractor++}, an utility for detecting and measuring sources --- stars, galaxies --- in astronomical images.\\
The software is written mainly in C++14, with some wrapper code in Python. I am used to
profiling the software to identify performance bottlenecks and excessive memory utilization.\\
}

\cventry{June 2015 -- February 2018}{FTS Project Lead}{CERN (European Organization for Nuclear Research)}{Geneva}{Switzerland}
{
  The File Transfer Service (FTS) is responsible for the transfer scheduling from
  three LHC experiments, moving more than 15 PB of data per week. 
  My responsibilities included:\\
  \textbf{Project Manager}
  \begin{itemize}
    \item Organize meetings with stakeholders
    \item Decide the direction and priorities
    \item Coordinate and oversee the global infrastructure
    \item Manage and improve the project life cycle, including the establishment
      of a QA process and guidelines
  \end{itemize}
  \textbf{Lead Developer}
  \begin{itemize}
    \item Maintenance of the core service, and protocol libraries
    \item Interact with other IT teams in order to
      improve the service level, and provide better feedback to our users via monitoring
  \end{itemize}
}

\cventry{October 2012 - May 2015}{Senior Software Engineer (Staff)}{CERN}{Geneva}{Switzerland}
{
  \begin{itemize}
    \item Design and implementation of the FTS REST API
    \item Development of an integrated Web Monitoring
    \item MySQL query optimization
    \item Consolidation of test tools and procedures
  \end{itemize}
}

\cventry{December 2010 -- September 2012}{Software Engineer (Fellow)}{CERN}{Geneva}{Switzerland}
{
  \begin{itemize}
    \item Re-factorization of `LCGDM/DPM', one of the storage components of the
      LHC Computing Grid in C++, increasing the system flexibility and code re-usability
    \item Development and maintenance of an Apache module for HTTP/WebDAV access
  \end{itemize}
}
\cventry{August 2009 -- November 2010}{Technical student / Associated User}{CERN}{Geneva}{Switzerland}
{
  \begin{itemize}
    \item Improve the Continuous Integration process for the Grid components
    \item Integration, testing and certification of Data Management components
  \end{itemize}
}
\cventry{February 2009 -- July 2009}{Teaching assistant}{Department of Computer Languages and Systems}{University of Cadiz}{Cadiz (Spain)}
	{Data Structures I (C) and Object Oriented Programming (C++)}
%\cventry{June 2008 -- August 2008}{CERN Openlab Summer Student}{CERN (European Organization for Nuclear Research)}{Geneva}{Switzerland}
%	{Maintenance and performance improvements of the web application used
%  by the CERN School of Computing to manage applicants.}
%\cventry{October 2007 -- June 2008}{Stagier}{Free Software and Open Content Office of University of Cadiz}{Cadiz}{}{System administrator of the main server.}
%\cventry{October 2006 -- February 2007}{Research assistant}{Free Software Office of University of Cadiz}{Cadiz}{}{Development
%of an open standards report for the Andalusia Regional Government.}


\section{Education}
\cventry{2017 -- *}{PhD Student on Computer Engineering}{University of Cadiz}{}{}{}
\cventry{2007 -- 2010}{Computer Engineering}{University of Cadiz}{equivalent to a Master's degree}{}{
Master Thesis: Takuan: Dynamic Invariant Generation System for WS-BPEL Compositions
}
\cventry{2003 -- 2007}{Computer Technical Engineering}{University of Cadiz}{}{}{}

\section{Languages}
\cventry{Spanish}{Mother tongue}{}{}{}{}
\cventry{English}{Advanced}{}{}{}{}
\cventry{French}{Intermediate}{}{}{}{}

\section{Training}
\cventry{2017}{8th International SuperComputing Camp}{University of Cádiz}{}{}{}
%\cventry{2017}{Machine Learning Foundations}{Coursera}{}{}{}
\cventry{2015}{Personal Awareness \& Impact}{CERN Training}{}{}{}
\cventry{2015}{Agile Project Management with Scrum}{CERN Training}{}{}{}
%\cventry{2014}{Making presentations}{CERN Training}{}{}{}
\cventry{2014}{Cryptography I}{Coursera}{}{}{}
\cventry{2013}{Communicating Effectively}{CERN Training}{}{}{}
\cventry{2013}{Hands On Modern C++: Making the most of the 11/14 standards}{CERN Training}{}{}{}
\cventry{2013}{Thematic CERN School of Computing}{Split}{Croatia}{}{Data oriented design, Memory programming, Parallelism, Efficient computing, Acceleration}
\cventry{2011}{CERN School of Computing}{Copenhagen}{Denmark}{}{}
%\cventry{2011}{Python - Hands-on Introduction}{CERN Training}{}{}{}
\cventry{2010}{Secure coding in C/C++}{CERN Training}{}{}{}


\newpage
\section{Project details}

\cventry{2018-*}{Euclid Photo-Z Pipeline}{}{}{}{
Euclid is a space telescope to improve our understanding of dark energy and dark matter.
An integral part of achieving its objectives is determining billions of galaxies'
red-shift (i.e., distance). The Photo-Z team designs and develops algorithms in C++14 and Python to this end. \\
\textbf{Keywords:} C++14, Python, pybind11, Optimizations, Sanitizers (AddressSanitizer, ThreadSanitizer), Coverity
}

\cventry{2018-*}{SourceXtractor++}{}{}{}{
\href{https://github.com/astrorama/SourceXtractorPlusPlus}{SourceXtractor++}
is a program that extracts a catalog of sources from astronomical images.
It is the successor to the original SExtractor package, completely rewritten in C++, with a
modular code design with support for third-party plug-ins. It adds support to a flexible
model fitting that can be configured in Python. To avoid the performance penalty 
of calling Python from a tight C++ loop, the model is
translated to an expression tree using a small custom library called
\href{https://github.com/astrorama/Alexandria/tree/develop/Pyston}{Pyston}.\\
\textbf{Keywords:} C++14, Python, Optimizations, Sanitizers (AddressSanitizer, ThreadSanitizer), Coverity,
Fedora packaging
}

\cventry{2012-2018}{File Transfer Service}{}{}{}{
The File Transfer Service (FTS) is responsible for the transfer scheduling the
LHC experiments ATLAS, CMS and LHCb.
During my period as a lead, this service moved 15PB of data (23M files) per week.\\
\textbf{Keywords:} C++11, Python, MySQL, Network protocols, Golang, ZeroMQ, STOMP, Kibana, Coverity
}

\cventry{2012-2018}{GFAL2}{}{}{}{
The Grid File Access Library is a C library providing an abstraction layer of
the grid storage system complexity.\\
It is used by FTS3, the experiment frameworks, and directly by end users.\\
\textbf{Keywords:} C, C++11, Python, Network protocols, Coverity
}

\cventry{2010-2012}{Disk Pool Manager}{}{}{}{
The Disk Pool Manager (DPM) is a storage system for grid sites.
It offers a simple way to create a disk-based grid storage element composed by
many disk servers.\\
DPM supports the data and metadata access protocols that are relevant for
file management and access in the Grid environment.\\
It was the most deployed storage solution on the LHC Computing Grid with 141 instances.
\textbf{Keywords:} C, C++03, MySQL, HTTP, WebDAV, Python, Network protocols
}

%\cventry{2010-2012}{SAKET}{\textit{Swiss Army Knife for ETICS Testing}}{}{}{
%SAKET was a tool written in Python that automated the execution of builds,
%deployment and tests in ETICS, a web application to build and test the software
%components within the EMI project.\\
%It was used by the \textit{Data Management Systems} section (\textit{Grid Technology} group)
%for the nightly builds, and also for maintaining an archive of these builds and tests.\\
%\textbf{Keywords:} Python, Continuous Integration, SVN.
%}

\cventry{2009-2010}{Takuan}{}{}{}{
Takuan is an open-source WS-BPEL dynamic invariant generator which can infer
invariants from WS-BPEL process definitions by generalizing from a set of
samples of their data and control flows.\\
As part of my Master Thesis, I improved the support of WS-BPEL characteristics
(as properties); I developed a graphical user interface to facilitate running
the tool; and added the capability to obtain instruction, branch and path coverage
from the execution of a test-suite.\\
\textbf{Original title:} \href{http://rodin.uca.es/xmlui/bitstream/handle/10498/8940/pfc.pdf}{Reingeniería y ampliación del generador dinámico
de invariantes potenciales para composiciones de servicios web en WS-BPEL Takuan}\\
\textbf{Keywords:} Java, XSLT, WS-BPEL, code coverage, Perl
}

%\newpage
\section{Papers / Lectures}
\cventry{Poster}{Deploying FTS with Docker Swarm and Openstack Magnum}
{Computing in High Energy Physics 2016}{October 2016}{San Francisco, United States}
{A. Alvarez Ayllon, A. Manzi, M. Arsuaga Ríos, R. Brito da Rocha}

\cventry{Paper}{Integration of end-user Cloud storage for CMS Analysis}
{Cloud Services for Synchronisation and Sharing}{January 2016}{}
{H. Riahi, A. Aimar, A. Álvarez Ayllón, J. Balcas, \ldots}

\cventry{Paper}{FTS3/WebFTS - A Powerful File Transfer Service for Scientific Communities}
{Procedia Computer Science}{December 2015}{}
{A. Kiryanov, A. Álvarez Ayllón, O. Keeble}

\cventry{Paper}{FTS3: New Data Movement Service for WLCG}
{Journal of Physics Conference Series 513}{June 2014}{}
{A. Álvarez Ayllón, M. Salichos, M.K. Simon, O. Keeble}

\cventry{Master thesis}{Re-engineering and improvement of the dynamic potential invariants generation for web services compositions in WS-BPEL Takuan}{}
	{Takuan is a framework that generates invariants of WS-BPEL (\textit{Web Services - Business Process Execution Language}) compositions using white-box testing}
  {}{}

\cventry{Poster}{Web Enabled Data Management with DPM \& LFC}
{Computing in High Energy Physics 2012}{May 2012}{New York, United States}
{A. Álvarez Ayllón, A. Beche, F. Furano, M. Hellmich, O. Keeble, R. Brito da Rocha}

\cventry{Poster}{DPM: Future Proof Storage}
{Computing in High Energy Physics 2012}{May 2012}{New York, United States}
{A. Álvarez Ayllón, A. Beche, F. Furano, M. Hellmich, O. Keeble, R. Brito da Rocha}

\cventry{Speaker}{The gLite Data Management Continuous Integration and Testing Process}
{1st EMI Technical Conference}{April 2011}{Vilnius, Lithuania}
{}{}

\cventry{Paper}{Takuan: A Tool for WS-BPEL Composition Testing using Dynamic Invariant Generation}
{10th International Conference on Web Engineering}{July 2010}{}
{M. Palomo-Duarte, A. García-Domínguez, I. Medina-Bulo, A. Alvarez Ayllón, and J. Santacruz}

\cventry{Paper}{Takuan: dynamic invariant generation for web services compositions with WS-BPEL}
{XIV Jornadas de Ingeniería del Software y Bases de Datos}{Septiembre 2009}{}
{A. García Domínguez, M. Palomo Duarte, I. Medina Bulo and A. Álvarez Ayllon}

\cventry{Paper}{The test cases in the dynamic invariant generation for web services compositions with WS-BPEL}
{V Jornadas Científico Técnicas en Servicios Web y SOA}{Septiembre 2009}{}
{A. Álvarez Ayllón, A. García Domínguez, M. Palomo Duarte and I. Medina Bulo}

\cventry{Paper}{The test cases coverage in the dynamic invariant generation for WS-BPEL compositions}
{IV Taller sobre Pruebas en Ingeniería del Software}{Septiembre 2009}{}
{M. Palomo Duarte, A. Álvarez Ayllón, A. García Domínguez and I. Medina Bulo}

\cventry{Paper}{Experiences in the development of an open standards taxonomy}
{Included in the Proceedings of the \textit{FLOSS International Conference} (Peer-reviewed)}{ISBN: 978-84-9828-124-8}{Pages 280 to 293}
{G. Aburruzaga García, A. Álvarez Ayllón, A. García Domínguez, J.C. González Cerezo, M. Palomo Duarte and J.R. Rodríguez Galván}

\closesection{}                   % needed to renewcommands
\renewcommand{\listitemsymbol}{-} % change the symbol for lists

\end{document}
